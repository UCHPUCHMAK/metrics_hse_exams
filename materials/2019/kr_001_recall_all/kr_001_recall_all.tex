\documentclass[12pt]{article}

\usepackage{tikz} % картинки в tikz
\usepackage{microtype} % свешивание пунктуации

\usepackage{array} % для столбцов фиксированной ширины

\usepackage{indentfirst} % отступ в первом параграфе

\usepackage{sectsty} % для центрирования названий частей
\allsectionsfont{\centering}

\usepackage{amsmath} % куча стандартных математических плюшек

\usepackage{comment}

\usepackage[top=2cm, left=1.2cm, right=1.2cm, bottom=2cm]{geometry} % размер текста на странице

\usepackage{lastpage} % чтобы узнать номер последней страницы

\usepackage{enumitem} % дополнительные плюшки для списков
%  например \begin{enumerate}[resume] позволяет продолжить нумерацию в новом списке
\usepackage{caption}


\usepackage{fancyhdr} % весёлые колонтитулы
\pagestyle{fancy}
\lhead{Эконометрика}
\chead{}
\rhead{2019-09-06, праздник «Вспомнить всё»}
\lfoot{}
\cfoot{}
\rfoot{\thepage/\pageref{LastPage}}
\renewcommand{\headrulewidth}{0.4pt}
\renewcommand{\footrulewidth}{0.4pt}



\usepackage{todonotes} % для вставки в документ заметок о том, что осталось сделать
% \todo{Здесь надо коэффициенты исправить}
% \missingfigure{Здесь будет Последний день Помпеи}
% \listoftodos - печатает все поставленные \todo'шки


% более красивые таблицы
\usepackage{booktabs}
% заповеди из докупентации:
% 1. Не используйте вертикальные линни
% 2. Не используйте двойные линии
% 3. Единицы измерения - в шапку таблицы
% 4. Не сокращайте .1 вместо 0.1
% 5. Повторяющееся значение повторяйте, а не говорите "то же"



\usepackage{fontspec}
\usepackage{polyglossia}

\setmainlanguage{russian}
\setotherlanguages{english}

% download "Linux Libertine" fonts:
% http://www.linuxlibertine.org/index.php?id=91&L=1
\setmainfont{Linux Libertine O} % or Helvetica, Arial, Cambria
% why do we need \newfontfamily:
% http://tex.stackexchange.com/questions/91507/
\newfontfamily{\cyrillicfonttt}{Linux Libertine O}

\AddEnumerateCounter{\asbuk}{\russian@alph}{щ} % для списков с русскими буквами
\setlist[enumerate, 2]{label=\asbuk*),ref=\asbuk*}

%% эконометрические сокращения
\DeclareMathOperator{\Cov}{\mathbb{C}ov}
\DeclareMathOperator{\Corr}{\mathbb{C}orr}
\DeclareMathOperator{\Var}{\mathbb{V}ar}
\DeclareMathOperator{\E}{\mathbb{E}}
\DeclareMathOperator{\tr}{trace}
\def \hb{\hat{\beta}}
\def \hs{\hat{\sigma}}
\def \htheta{\hat{\theta}}
\def \s{\sigma}
\def \hy{\hat{y}}
\def \hY{\hat{Y}}
\def \v1{\vec{1}}
\def \e{\varepsilon}
\def \he{\hat{\e}}
\def \z{z}
\def \hVar{\widehat{\Var}}
\def \hCorr{\widehat{\Corr}}
\def \hCov{\widehat{\Cov}}
\def \cN{\mathcal{N}}


\begin{document}
 



\begin{enumerate}

  \item Сфорулируйте теорему о трёх перпендикулярах и обратную к ней. Нарисуйте картинку.

  \item Для матрицы
$
  A=\begin{pmatrix}
  7 & 5  \\
  5 & 7  \\
  \end{pmatrix}
$

  \begin{enumerate}
  \item Найдите собственные числа и собственные векторы матрицы;
  \item Найдите определитель $\det A$ и след $\tr A$;
 \item Известно, что $B = 2019A^{-1} + 2018I$, где $I$ — единичная матрица.
 Найдите собственные числа $B$, определитель $\det B$ и след $\tr B$.

  \end{enumerate}


  \item Блондинка Маша встретила 200 динозавров.
  Средний рост динозавров оказался равен 20 метров, а выборочное стандартное отклонение — 5 метров.

  \begin{enumerate}
    \item Постройте 95\% доверительный интервал для математического ожидания роста динозавра.
    \item На уровне значимости 1\% проверьте гипотезу о том, что математическое ожидание
    роста равно 22 метрам. Против альтернативной гипотезе о неравенстве.
    \item Укажите $P$-значение для теста в предыдущем пункте.
  \end{enumerate}

 \item В убийстве равновероятно виноват либо Джон, либо Билл. 
 На месте убийства найдена кровь убийцы, совпадающая с группой крови Джона. Такой группой крови обладает 10\% населения. Группа крови Билла неизвестна. 
 Какова услованя вероятность того, что Билл — убийца?

\end{enumerate}



\newpage


\begin{enumerate}

  \item Сфорулируйте теорему о трёх перпендикулярах и обратную к ней. Нарисуйте картинку.

  \item Для матрицы
$
  A=\begin{pmatrix}
  7 & 5  \\
  5 & 7  \\
  \end{pmatrix}
$

  \begin{enumerate}
  \item Найдите собственные числа и собственные векторы матрицы;
  \item Найдите определитель $\det A$ и след $\tr A$;
 \item Известно, что $B = 2019A^{-1} + 2018I$, где $I$ — единичная матрица.
 Найдите собственные числа $B$, определитель $\det B$ и след $\tr B$.

  \end{enumerate}


  \item Блондинка Маша встретила 200 динозавров.
  Средний рост динозавров оказался равен 20 метров, а выборочное стандартное отклонение — 5 метров.

  \begin{enumerate}
    \item Постройте 95\% доверительный интервал для математического ожидания роста динозавра.
    \item На уровне значимости 1\% проверьте гипотезу о том, что математическое ожидание
    роста равно 22 метрам. Против альтернативной гипотезе о неравенстве.
    \item Укажите $P$-значение для теста в предыдущем пункте.
  \end{enumerate}

 \item В убийстве равновероятно виноват либо Джон, либо Билл. 
 На месте убийства найдена кровь убийцы, совпадающая с группой крови Джона. Такой группой крови обладает 10\% населения. Группа крови Билла неизвестна. 
 Какова услованя вероятность того, что Билл — убийца?

\end{enumerate}



\end{document}
