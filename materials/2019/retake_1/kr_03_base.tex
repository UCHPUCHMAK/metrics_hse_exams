\documentclass[12pt]{article}

\usepackage{tikz} % картинки в tikz
\usepackage{microtype} % свешивание пунктуации

\usepackage{array} % для столбцов фиксированной ширины


\usepackage{comment} % для комментирования целых окружений

\usepackage{indentfirst} % отступ в первом параграфе

\usepackage{sectsty} % для центрирования названий частей
\allsectionsfont{\centering}

\usepackage{amsmath, amssymb, amsthm} % куча стандартных математических плюшек

\usepackage{amsfonts}

\usepackage[top=2cm, left=1cm, right=1cm, bottom=2cm]{geometry} % размер текста на странице

\usepackage{lastpage} % чтобы узнать номер последней страницы

\usepackage{enumitem} % дополнительные плюшки для списков
%  например \begin{enumerate}[resume] позволяет продолжить нумерацию в новом списке
\usepackage{caption}

\usepackage{hyperref} % гиперссылки

\usepackage{multicol} % текст в несколько столбцов


\usepackage{fancyhdr} % весёлые колонтитулы
\pagestyle{fancy}
\lhead{Эконометрика, ВШЭ}
\chead{Доп задачи}
\rhead{2019-09-25}
\lfoot{Вариант $\mu$}
\cfoot{Паниковать запрещается!}
\rfoot{Тест}
\renewcommand{\headrulewidth}{0.4pt}
\renewcommand{\footrulewidth}{0.4pt}

\usepackage{ifthen} % для написания условий

\usepackage{todonotes} % для вставки в документ заметок о том, что осталось сделать
% \todo{Здесь надо коэффициенты исправить}
% \missingfigure{Здесь будет Последний день Помпеи}
% \listoftodos --- печатает все поставленные \todo'шки


% более красивые таблицы
\usepackage{booktabs}
% заповеди из докупентации:
% 1. Не используйте вертикальные линни
% 2. Не используйте двойные линии
% 3. Единицы измерения - в шапку таблицы
% 4. Не сокращайте .1 вместо 0.1
% 5. Повторяющееся значение повторяйте, а не говорите "то же"


\usepackage{fontspec}
\usepackage{polyglossia}

\setmainlanguage{russian}
\setotherlanguages{english}

% download "Linux Libertine" fonts:
% http://www.linuxlibertine.org/index.php?id=91&L=1
\setmainfont{Linux Libertine O} % or Helvetica, Arial, Cambria
% why do we need \newfontfamily:
% http://tex.stackexchange.com/questions/91507/
\newfontfamily{\cyrillicfonttt}{Linux Libertine O}

\AddEnumerateCounter{\asbuk}{\russian@alph}{щ} % для списков с русскими буквами
\setlist[enumerate, 2]{label=\asbuk*),ref=\asbuk*}

%% эконометрические сокращения
\DeclareMathOperator{\Cov}{Cov}
\DeclareMathOperator{\Corr}{Corr}
\DeclareMathOperator{\Var}{Var}
\DeclareMathOperator{\E}{E}
\def \hb{\hat{\beta}}
\def \hs{\hat{\sigma}}
\def \htheta{\hat{\theta}}
\def \s{\sigma}
\def \hy{\hat{y}}
\def \hY{\hat{Y}}
\def \v1{\vec{1}}
\def \e{\varepsilon}
\def \he{\hat{\e}}
\def \z{z}
\def \hVar{\widehat{\Var}}
\def \hCorr{\widehat{\Corr}}
\def \hCov{\widehat{\Cov}}
\def \cN{\mathcal{N}}
\def \P{\mathbb{P}}


\def \putyourname{\fbox{
    \begin{minipage}{42em}
      Фамилия, имя, номер группы:\vspace*{3ex}\par
      \noindent\dotfill\vspace{2mm}
    \end{minipage}
  }
}

\def \checktable{

	\vspace{5pt}
	Табличка для проверяющих работу:

\vspace{5pt}

	\begin{tabular}{|m{2cm}|m{1cm}|m{1cm}|m{1cm}|m{1cm}|m{1cm}|m{2cm}|}
\toprule
		Тест & 1 &  2 & 3 & 4 & 5 & Итого \\
\midrule
		&  &  & & & & \\
		&  &  & & & & \\
 \bottomrule
\end{tabular}
}



\def \testtable{

	\vspace{5pt}
	Внесите сюда ответы на тест:

\vspace{5pt}

\begin{tabular}{|m{2cm}|m{0.6cm}|m{0.6cm}|m{0.6cm}|m{0.6cm}|m{0.6cm}|m{0.6cm}|m{0.6cm}|m{0.6cm}|m{0.6cm}|m{0.6cm}|}
\toprule
		Вопрос & 1 &  2 & 3 & 4 & 5 & 6 & 7 & 8 & 9 & 10 \\
\midrule
		Ответ &  &  & & & & & & & & \\
 \bottomrule
\end{tabular}
}



\begin{document}





\begin{enumerate}

\item
По 200 фирмам была оценена зависимость выпуска $Y$ от труда $L$ и капитала $K$
с помощью двух моделей:

Модель Кобба-Дугласа: $\ln{Y_i} = \beta_0 + \beta_1 \ln{L_i} + \beta_2 \ln{K_i} +
\varepsilon_i$

Транслоговая модель: $\ln{Y_i} = \gamma_0 + \gamma_1 \ln{L_i} + \gamma_2 \ln{K_i} +
\gamma_3 (0.5 \ln^2{L_i}) + \gamma_4 (0.5 \ln^2{K_i}) + \gamma_5 \ln{K_i} \ln{L_i} +
\varepsilon_i$

Оценки коэффициентов обеих моделей (в скобках приведены стандартные ошибки):

\begin{tabular}{lcc}
\toprule
Переменная & Модель Кобба-Дугласа & Транслоговая модель \\
\midrule
константа & $1.1706$ ($0.326$) & $0.9441$ ($2.911$)   \\
$\ln L$ & $0.6029$ ($0.125$) & $3.613$ ($1.548$)  \\
$\ln K$ & $0.375$ ($0.085$) & $-1.893$ ($1.016$)  \\
$0.5 \ln^2 L$ &  & $-0.964$ ($0.707$)  \\
$0.5 \ln^2 K$ & & $0.0852$ ($0.2922$) \\
$\ln L \ln K$ & & $0.3123$ ($0.4389$)  \\
$R^2$ & $0.9$ & $0.954$  \\
\bottomrule
\end{tabular}

В модели Кобба-Дугласа $\hCov(\hb_1, \hb_2)= 0.01$.

На уровне значимости $\alpha = 0.05$ проверьте следующие гипотезы:
\begin{enumerate}
\item В модели Кобба-Дугласа эластичность выпуска по капиталу равна единице.
\item В модели Кобба-Дугласа эластичности выпуска по труду и капиталу одинаковы.
\item В транслоговой модели $\gamma_3 = 0$.
\item В транслоговой модели $\gamma_3 = \gamma_4 = \gamma_5 = 0$.
\end{enumerate}



\item
Исследователь Д'Артаньян стандартизировал (центрировал и нормировал) все
имеющиеся регрессоры и поместил их в столбцы матрицы $\tilde X$. Выборочная
корреляционная матрица регрессоров равна:
\[
\begin{pmatrix}
1 & -0.85 & 0  \\
-0.85 & 1 & 0  \\
0 & 0 & 1 \\
\end{pmatrix}.
\]
\begin{enumerate}
    \item Найдите параметр обусловленности (condition number) матрицы
    $\tilde X^T \tilde X$.
    \item Вычислите одну или две главные компоненты, объясняющие не менее
    $70$\% суммарной дисперсии стандартизированных регрессоров. Выпишите
    найденные компоненты как линейные комбинации столбцов матрицы $\tilde X$.
\end{enumerate}

\newpage



\item Для регрессии в отклонениях $y = \beta_1 x + \beta_2 z + \varepsilon$, оцененной по 100 наблюдениям, известны следующие суммы:
\[
\sum y^2_i = \frac{493}{3}, \sum x_i^2 = 30, \sum z_i^2 = 3, \sum x_i y_i = 30, \sum z_i y_i = 20, \sum x_i z_i = 0
\]

Найдите оценки МНК коэффициентов $\beta_1, \beta_2$ и коэффициент детерминации $R^2$.


\item По ежегодным данным за 25 лет была оценена зависимость расходов на жилье $Y$ от доходов индивидуумов $I$ и относительного индекса цен $P$ с помощью трёх моделей:

$\hY = \underset{(46.9)}{-39.9} + \underset{(0.009)}{0.179} P + \underset{(0.409)}{0.113} I, R^2 = 0.987$

$\hY = \underset{(3.34)}{-27.03} + \underset{(0.004)}{0.177} I, R^2 = 0.987$

$\hY = \underset{(24.2)}{813.3} + \underset{(0.019)}{-7.08} P, R^2 = 0.76$

Известно, что $\hCorr (P, I) = -0.88$, в скобках указаны стандартные отклонения коэффициентов.

Какую модель Вы предпочтёте и почему?

\item Выпишите целевую функцию алгоритма LASSO. Объясните, как алгоритм LASSO заменяет проверку гипотез
о значимости коэффициентов в классическом МНК.


\end{enumerate}





\end{document}
