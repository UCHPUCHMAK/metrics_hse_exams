\documentclass[12pt]{article}

\usepackage{tikz} % картинки в tikz
\usepackage{microtype} % свешивание пунктуации

\usepackage{array} % для столбцов фиксированной ширины

\usepackage{indentfirst} % отступ в первом параграфе

\usepackage{sectsty} % для центрирования названий частей
\allsectionsfont{\centering}

\usepackage{amsmath, amssymb, amsthm} % куча стандартных математических плюшек

\usepackage{amsfonts}

\usepackage{comment}

\usepackage[top=2cm, left=1.2cm, right=1.2cm, bottom=2cm]{geometry} % размер текста на странице

\usepackage{lastpage} % чтобы узнать номер последней страницы

\usepackage{enumitem} % дополнительные плюшки для списков
%  например \begin{enumerate}[resume] позволяет продолжить нумерацию в новом списке
\usepackage{caption}


\usepackage{hyperref} % гиперссылки

\usepackage{multicol} % текст в несколько столбцов


\usepackage{fancyhdr} % весёлые колонтитулы
\pagestyle{fancy}
\lhead{Эконометрика, НИУ-ВШЭ}
\chead{контрольная работа №1}
\rhead{2019-10-19}
\lfoot{Вариант $\xi$}
\cfoot{Ни пуха, ни пера!}
\rfoot{\thepage/3}
\renewcommand{\headrulewidth}{0.4pt}
\renewcommand{\footrulewidth}{0.4pt}



\usepackage{todonotes} % для вставки в документ заметок о том, что осталось сделать
% \todo{Здесь надо коэффициенты исправить}
% \missingfigure{Здесь будет Последний день Помпеи}
% \listoftodos - печатает все поставленные \todo'шки


% более красивые таблицы
\usepackage{booktabs}
% заповеди из докупентации:
% 1. Не используйте вертикальные линни
% 2. Не используйте двойные линии
% 3. Единицы измерения - в шапку таблицы
% 4. Не сокращайте .1 вместо 0.1
% 5. Повторяющееся значение повторяйте, а не говорите "то же"



\usepackage{fontspec}
\usepackage{polyglossia}

\setmainlanguage{russian}
\setotherlanguages{english}

% download "Linux Libertine" fonts:
% http://www.linuxlibertine.org/index.php?id=91&L=1
\setmainfont{Linux Libertine O} % or Helvetica, Arial, Cambria
% why do we need \newfontfamily:
% http://tex.stackexchange.com/questions/91507/
\newfontfamily{\cyrillicfonttt}{Linux Libertine O}

\AddEnumerateCounter{\asbuk}{\russian@alph}{щ} % для списков с русскими буквами
\setlist[enumerate, 2]{label=\asbuk*),ref=\asbuk*}

%% эконометрические сокращения
\let\P\relax
\DeclareMathOperator{\Cov}{\mathbb{C}ov}
\DeclareMathOperator{\Corr}{\mathbb{C}orr}
\DeclareMathOperator{\Var}{\mathbb{V}ar}
\DeclareMathOperator{\E}{\mathbb{E}}
\DeclareMathOperator{\P}{\mathbb{P}}
\DeclareMathOperator{\tr}{trace}
\def \hb{\hat{\beta}}
\def \hs{\hat{\sigma}}
\def \htheta{\hat{\theta}}
\def \s{\sigma}
\def \hy{\hat{y}}
\def \hY{\hat{Y}}
\def \v1{\vec{1}}
\def \e{\varepsilon}
\def \he{\hat{\e}}
\def \z{z}
\def \hVar{\widehat{\Var}}
\def \hCorr{\widehat{\Corr}}
\def \hCov{\widehat{\Cov}}
\def \cN{\mathcal{N}}





\def \putyourname{\fbox{
    \begin{minipage}{42em}
      Фамилия, имя, номер группы:\vspace*{3ex}\par
      \noindent\dotfill\vspace{2mm}
    \end{minipage}
  }
}

\def \checktable{
\begin{minipage}{42em}
\begin{tabular}{|m{2cm}|m{2cm}|m{2cm}|m{2cm}|m{2cm}|}
\hline
Тест & 1 &  2 & 3 & Итого \\ \hline
&  &  &  & \\
 &  &   & & \\
 \hline
\end{tabular} $\longleftarrow$ для проверяющего!
\end{minipage}
}

\def \testtable{
\begin{minipage}{42em}
\vspace{4pt}

Ответы на тест:

\vspace{2pt}
\begin{tabular}{|m{1cm}|m{1cm}|m{1cm}|m{1cm}|m{1cm}|m{1cm}|m{1cm}|m{1cm}|m{1cm}|m{1cm}|}
\hline
1 & 2 &  3 & 4 & 5 & 6 & 7 & 8 & 9 & 10 \\ 
\hline
 &  &   &  &  &  &  &  &  &  \\ 
 &  &   &  &  &  &  &  &  &  \\ 
\hline
\end{tabular}
\end{minipage}

}





% [1][3] 1 = one argument, 3 = value if missing
% эта магия создаёт окружение answerlist
% именно в окружении answerlist записаны варианты ответов в подключаемых exerciseXX
% просто \begin{answerlist} сделает ответы в три столбца
% если ответы длинные, то надо в них руками сделать
% \begin{answerlist}[1] чтобы они шли в один столбец
\newenvironment{answerlist}[1][3]{
\begin{multicols}{#1}
\begin{enumerate}[label=\fbox{\emph{\Alph*}},ref=\emph{\alph*}]
}
{
\end{enumerate}
\end{multicols}
}

\newenvironment{answerlist1}{
\begin{enumerate}[label=\fbox{\emph{\Alph*}},ref=\emph{\alph*}]
}
{
\end{enumerate}
}



\excludecomment{solution} % without solutions

\theoremstyle{definition}
\newtheorem{question}{Вопрос}




\begin{document}

\checktable

\putyourname

\testtable

\subsection*{Тест}

\begin{question}
Случайные величины $X$ и $Y$ независимы и имеют нормальное распределение с $\E(X) = 0$, $\Var(X)=1$, $\E(Y)=5$, $\Var(Y) = 4$. 
Величина $Z = 2X + Y$ имеет распределение
\begin{answerlist}
  \item $\cN(5;5)$
  \item $\cN(5;8)$
  \item $\chi^2_2$
  \item $t_2$
  \item $F_{1,1}$
  \item нет верного ответа
\end{answerlist}
\end{question}


\begin{question}
Оценка $T_n = T(X_1, X_2, \ldots, X_n)$ называется несмещённой оценкой параметра $\theta$, если
\begin{answerlist}[2]
  \item $\E(T_n) = T_n$
  \item $T_n = 0$
  \item $\lim_{n\to\infty} \P(|T_n - \theta|>\e) = 0$ при $\e>0$
  \item $\E(T_n) = 0$
  \item $\E(T_n) = \theta$
  \item нет верного ответа
\end{answerlist}
\end{question}



\begin{question}
Оценена регрессия $\hat Y = 300 + 6W$, где $R^2 = 0.85$ и $W_i = X_i / X_{i-1}$.

Если объясняющая переменная будет выражена в процентах, $\tilde W_i = 100(X_i - X_{i-1})/X_{i-1}$, то результаты оценки регрессии примут вид
\begin{answerlist}
  \item $\hat Y_i = 3 + 6 \tilde W_i$, $R^2= 0.85$
  \item $\hat Y_i = 300 + 600 \tilde W_i$, $R^2= 0.85$
  \item $\hat Y_i = 306 + 0.06 \tilde W_i$, $R^2= 0.85$
  \item $\hat Y_i = 300 + 6 \tilde W_i$, $R^2= 0.085$
  \item $\hat Y_i = 300 + 6 \tilde W_i$, $R^2= 0.85$
  \item нет верного ответа
\end{answerlist}
\end{question}


\begin{question}
Оценка ковариационной матрицы оценок коэффициентов регрессии $Y=X\beta + \e$ пропорциональна
\begin{answerlist}
  \item $(XX^T)^{-1}$
  \item $X^TX$
  \item $(X^TX)^{-1}$
  \item $XX^T$
  \item $X^TY$
  \item нет верного ответа
\end{answerlist}
\end{question}


\newpage
\begin{question}
Среди предпосылок теоремы Гаусса-Маркова фигурирует условие
\begin{answerlist}
  \item $\E(Y_i)=0$
  \item $\e_i \sim \cN(0;\sigma^2)$
  \item $\E(\e_i)=1$
  \item $\Var(\e_i)=const$
  \item $\Var(\e_i)=1$
  \item нет верного ответа
\end{answerlist}
\end{question}


\begin{question}
Оценено уравнение парной регрессии $Y_i = \beta_0 + \beta_1 X_i + \e_i$, причём МНК-оценка
коэффициента $\beta_1$ равна 5, а стандартная ошибка оценки равна $0.25$.

Значение $t$-статистики для проверки гипотезы, что этот коэффициент равен 4, есть
\begin{answerlist}
  \item $-2$
  \item $4$
  \item $-4$
  \item $2$
  \item $20$
  \item нет верного ответа
\end{answerlist}
\end{question}

\begin{question}
P-значение при проверке некоторой гипотезы $H_0$ оказалось равно $0.002$.

Гипотеза $H_0$ не отвергается при уровне значимости
\begin{answerlist}
    \item 10\%
    \item 0.1\%  
    \item 1\%
  \item 5\%
  \item всех перечисленных
  \item нет верного ответа
\end{answerlist}
\end{question}


\begin{question}
Известно, что выборочный коэффициент корреляции между $X$ и $Y$ равен $0.25$. 
В регрессии $Y$ на константу и $X$ коэффициент $R^2$ равен
\begin{answerlist}
  \item $25$
  \item $0.25$
  \item $0.5$
  \item $0.0625$
  \item $\sqrt{0.5}$
  \item нет верного ответа
\end{answerlist}
\end{question}



\begin{question}
Исследователь оценил регрессию $\hat Y_i = 90 + 3 X_i$. Если увеличить переменную $X$ на 10\%,
а $Y$ — на 10 единиц, то 

\begin{answerlist}[2]
  \item оценка коэффициента $\beta_0$ уменьшится, а $\beta_1$ — увеличится
  \item оценка коэффициента $\beta_0$ увеличится, а $\beta_1$ — уменьшится
  \item оценки коэффициентов $\beta_0$, $\beta_1$ не изменятся
  \item оценки коэффициентов $\beta_0$, $\beta_1$ уменьшатся
  \item оценки коэффициентов $\beta_0$, $\beta_1$ увеличатся
  \item нет верного ответа
\end{answerlist}
\end{question}

\begin{question}
Исследователь оценил регрессию $\hat Y_i = \underset{(0.1)}{30} + \underset{(0.5)}{6} X_i$, причём $\sum_i (X_i - \bar X)^2=4$. Все предпосылки теоремы Гаусса-Маркова выполнены. 

В скобках приведены стандартные ошибки коэффициентов. 
Несмещённая оценка дисперсии ошибок регрессии равна 
\begin{answerlist}
  \item $0.25$
  \item $2$
  \item $1$
  \item $0.125$
  \item $2\sqrt{0.5}$
  \item нет верного ответа
\end{answerlist}
\end{question}


\newpage
\putyourname

\subsection*{Задачи}

\begin{enumerate}
\item Найдите величины Q1, \ldots, Q10, пропущенные в таблицах: 

\begin{tabular}{lr} \toprule
Indicator & Value \\
\midrule
Multiple R          & Q1 \\
$R^2$     			& Q2 \\
Adjusted $R^2$     	& 0.54 \\
Standart error 		& Q3 \\
Observations		& 800 \\
\bottomrule
\end{tabular}\hspace{2cm}
\begin{tabular}{lrrrrr} \toprule
ANOVA     	 &  df 	& SS		& MS 	& F & Significance F \\
\midrule
Regression   & Q4   	& 42.9  	& 42.9	&  923 	& 	0	\\
Residual     & 798  	& 37.0  	& 46	&  	&     	\\
Total        & 799  	& Q5        &    	&  	&     	\\
\bottomrule
\end{tabular}


\begin{tabular}{rrrrrrr}
\toprule
 			& Coef. 	& St. error	& t-stat & P-value	& Lower 95\% 	& Upper 95\% \\
\midrule
Intercept 	& -25.24 	& 2.0 	& Q6 		& 0 	&  Q7		& -21.31 \\
totspan		& 1.7		& Q8    & 30.4 	    & 0 	&  Q9	    & Q10 \\
\bottomrule
\end{tabular}


\item Грета Тунберг оценила зависимость средней температуры на Земном шаре в градусах, $Y_i$, 
от количества своих постов в твиттере в соответствующий день, $X_i$, по 52 дням:

\[
\hat Y_i = \underset{(1.24)}{-1.53} + \underset{(0.12)}{0.14} X_i, \text{ где } \sum_i (X_i - \bar X)^2 =52.4 \text{ и } \bar X = 10
\]

\begin{enumerate}
\item (2 балла) Проверьте гипотезы о незначимости каждого коэффициента при уровне значимости $\alpha = 0.01$.
\item (2 балла) Проверьте гипотезу о равенстве углового коэффициента 2 при альтернативной гипотезе, что 
коэффициент больше 2 и уровне значимости $\alpha = 0.01$.
\item (1 балл) Найдите оценку дисперсии $\e_i$ в модели $Y_i = \beta_0 + \beta_1 X_i + \e_i$.
\item (3 балла) Постройте 95\%-ый доверительный интервал для индивидуального прогноза $Y$, если $X=10$.


\end{enumerate}




\item Рассмотрим парную регрессию $\hat Y_i = \hat\beta_0 + \hat\beta_1 X_i$.

\begin{enumerate}
\item (1 балл) Дайте определение коэффициента детерминации $R^2$.
\item (1 + 2 балла) В каких пределах может лежать $R^2$ в указанной парной регрессии? Докажите сформулированное утверждение.
\item (1 + 2 балла) Как связан коэффициент $R^2$ и выборочная корреляция зависимой переменной и регрессора? Докажите сформулированное утверждение. 
\end{enumerate}

% Примечание: при доказательстве можно использовать условия первого порядка, важно их аккуратно сформулировать. 



\end{enumerate}



\end{document}