\documentclass[12pt]{article}

\usepackage{tikz} % картинки в tikz
\usepackage{microtype} % свешивание пунктуации

\usepackage{array} % для столбцов фиксированной ширины


\usepackage{comment} % для комментирования целых окружений

\usepackage{indentfirst} % отступ в первом параграфе

\usepackage{sectsty} % для центрирования названий частей
\allsectionsfont{\centering}

\usepackage{amsmath, amssymb, amsthm} % куча стандартных математических плюшек

\usepackage{amsfonts}

\usepackage[top=2cm, left=1cm, right=1cm, bottom=2cm]{geometry} % размер текста на странице

\usepackage{lastpage} % чтобы узнать номер последней страницы

\usepackage{enumitem} % дополнительные плюшки для списков
%  например \begin{enumerate}[resume] позволяет продолжить нумерацию в новом списке
\usepackage{caption}

\usepackage{hyperref} % гиперссылки

\usepackage{multicol} % текст в несколько столбцов


\usepackage{fancyhdr} % весёлые колонтитулы
\pagestyle{fancy}
\lhead{Эконометрика, ВШЭ}
\chead{Контрольная 1}
\rhead{2018-10-26}
\lfoot{Вариант $\mu$}
\cfoot{Паниковать запрещается!}
\rfoot{Тест}
\renewcommand{\headrulewidth}{0.4pt}
\renewcommand{\footrulewidth}{0.4pt}


 
\usepackage{todonotes} % для вставки в документ заметок о том, что осталось сделать
% \todo{Здесь надо коэффициенты исправить}
% \missingfigure{Здесь будет Последний день Помпеи}
% \listoftodos --- печатает все поставленные \todo'шки


% более красивые таблицы
\usepackage{booktabs}
% заповеди из докупентации:
% 1. Не используйте вертикальные линни
% 2. Не используйте двойные линии
% 3. Единицы измерения - в шапку таблицы
% 4. Не сокращайте .1 вместо 0.1
% 5. Повторяющееся значение повторяйте, а не говорите "то же"


\usepackage{fontspec}
\usepackage{polyglossia}

\setmainlanguage{russian}
\setotherlanguages{english}

% download "Linux Libertine" fonts:
% http://www.linuxlibertine.org/index.php?id=91&L=1
\setmainfont{Linux Libertine O} % or Helvetica, Arial, Cambria
% why do we need \newfontfamily:
% http://tex.stackexchange.com/questions/91507/
\newfontfamily{\cyrillicfonttt}{Linux Libertine O}

\AddEnumerateCounter{\asbuk}{\russian@alph}{щ} % для списков с русскими буквами
\setlist[enumerate, 2]{label=\asbuk*),ref=\asbuk*}

%% эконометрические сокращения
\DeclareMathOperator{\Cov}{Cov}
\DeclareMathOperator{\Corr}{Corr}
\DeclareMathOperator{\Var}{Var}
\DeclareMathOperator{\E}{E}
\def \hb{\hat{\beta}}
\def \hs{\hat{\sigma}}
\def \htheta{\hat{\theta}}
\def \s{\sigma}
\def \hy{\hat{y}}
\def \hY{\hat{Y}}
\def \v1{\vec{1}}
\def \e{\varepsilon}
\def \he{\hat{\e}}
\def \z{z}
\def \hVar{\widehat{\Var}}
\def \hCorr{\widehat{\Corr}}
\def \hCov{\widehat{\Cov}}
\def \cN{\mathcal{N}}
\def \P{\mathbb{P}}


\def \putyourname{\fbox{
    \begin{minipage}{42em}
      Фамилия, имя, номер группы:\vspace*{3ex}\par
      \noindent\dotfill\vspace{2mm}
    \end{minipage}
  }
}

\def \checktable{
\begin{table}[]
\begin{tabular}{|m{2cm}|m{2cm}|m{2cm}|m{2cm}|m{2cm}|m{2cm}|}
\hline
Тест & 1 &  2 & 3 & 4 & Итого \\ \hline
&  &  & & & \\
 &  &  & & & \\
 \hline
\end{tabular}
\end{table}
}


% [1][3] 1 = one argument, 3 = value if missing
% эта магия создаёт окружение answerlist
% именно в окружении answerlist записаны варианты ответов в подключаемых exerciseXX
% просто \begin{answerlist} сделает ответы в три столбца
% если ответы длинные, то надо в них руками сделать
% \begin{answerlist}[1] чтобы они шли в один столбец
\newenvironment{answerlist}[1][3]{
\begin{multicols}{#1}
\begin{enumerate}[label=\fbox{\emph{\Alph*}},ref=\emph{\alph*}]
}
{
\end{enumerate}
\end{multicols}
}

\excludecomment{solution} % without solutions

\theoremstyle{definition}
\newtheorem{question}{Вопрос}



\begin{document}

\putyourname

\begin{question}
При выполненных условиях регулярности оценки метода максимального
правдоподобия могут \textbf{НЕ} являться
\begin{answerlist}
  \item инвариантными
  \item асимптотически нормальными
  \item асимптотически эффективными
  \item состоятельными
  \item несмещёнными
\end{answerlist}
\end{question}




\begin{question}
Стьюдентизированные остатки регрессии используются
\begin{answerlist}
  \item на первом шаге при проведении теста Годфельда-Квандта
  \item в методе главных компонент
  \item на первом шаге двухшагового МНК
  \item в тесте Саргана
  \item для выявления выбросов
\end{answerlist}
\end{question}




\begin{question}
Рассмотрим логистическую регрессию с пятью регрессорами помимо
константы, оцениваемую методом максимального правдоподобия по \(n\)
наблюдениям. Cтатистика \(\hat \beta_3 / se(\hat\beta_3)\) для проверки
значимости коэффициента \(\beta_3\) имеет
\begin{answerlist}
  \item \(\chi^2\)-распределение с одной степенью свободы
  \item \(t\)-распределение с \(n\) степенями свободы
  \item \(t\)-распределение с \(n-6\) степенями свободы
  \item \(t\)-распределение с \(n-5\) степенями свободы
  \item асимптотически нормальное распределение
\end{answerlist}
\end{question}

\begin{solution}
========
\end{solution}



\begin{question}
Для модели \(Y_i = \beta X_i + \varepsilon_i\) c
\(\E(\varepsilon_i) = 0\) известно, что оценка
\(\hat \beta = \frac{\sum_{i=1}^n Y_i}{\sum_{i=1}^n X_i}\) обладает
наименьшей дисперсией среди линейных несмещённых оценок.

Дисперсии \(\Var(\varepsilon_i)\) пропорциональны
\begin{answerlist}
  \item \(\sqrt{X_i}\)
  \item \(X_i^2\)
  \item \(1/X_i\)
  \item \(X_i\)
  \item \(1/X_i^2\)
\end{answerlist}
\end{question}




\begin{question}
Основная гипотеза модели адаптивных ожиданий состоит в том, что
\begin{answerlist}[2]
  \item \(X_t^e - X_{t-1}^e = (1-\lambda) (X_t - X_{t-1}), \; 0 \leq \lambda < 1\)
  \item \(Y_t^e = \left(1 - \frac{1}{\delta}\right) Y_{t-1} + \frac{1}{\delta} Y_{t}, \; 0 < \delta \leq 1\)
  \item \(Y_t - Y_{t-1}^e = \delta (Y_t^e - Y_{t-1}), \; 0 < \delta \leq 1\)
  \item \(X_{t+1}^e - X_{t}^e = \lambda (X_{t} - X_{t}^e), \; 0 \leq \lambda < 1\)
  \item \(Y_t - Y_{t-1} = \delta (Y_t^e - Y_{t-1}), \; 0 < \delta \leq 1\)
\end{answerlist}
\end{question}





\newpage
\putyourname

\begin{question}
Использование скорректированных стандартных ошибок Уайта при
гомоскедастичности приведет к
\begin{answerlist}
  \item понижению эффективности МНК оценок коэффициентов
  \item смещённости МНК оценок коэффициентов
  \item несостоятельности МНК оценок коэффициентов
  \item получению состоятельной оценки дисперсии случайной ошибки
  \item повышению эффективности МНК оценок коэффициентов
\end{answerlist}
\end{question}

\begin{solution}
========
\end{solution}



\begin{question}
В линейной модели \(Y_i = \beta_0 + \beta_1 X_i + \varepsilon_i\)
стохастический регрессор и случайный член \(\varepsilon_i\)
коррелированы. Состоятельные оценки коэффициентов можно получить с
помощью
\begin{answerlist}
  \item обобщённого МНК
  \item метода наименьших квадратов
  \item взвешенного МНК
  \item метода главных компонент
  \item метода инструментальных переменных
\end{answerlist}
\end{question}




\begin{question}
В предположениях нормальности ошибок ширина 95\%-го интервала для
ожидаемого (среднего) значения \(Y_{n+1}\) равна \(1200\). Известно, что
\(\hat\sigma = 400\) и \(n=60\). Ширина 95\%-го интервала для
фактического (индивидального) значения \(Y_{n+1}\) примерно равна
\begin{answerlist}
  \item \(1500\)
  \item \(1600\)
  \item \(2000\)
  \item \(1000\)
  \item \(1400\)
\end{answerlist}
\end{question}

\begin{solution}
========
\end{solution}



\begin{question}
Рассмотрим модель
\(Y_i= \beta_0 + \beta_z Z_{i} + \beta_w W_{i} + \varepsilon\) при
гетероскедастичности. Стандартная ошибка МНК-оценки, рассчитываемая по
формуле \(se(\hat\beta_w)=\sqrt{RSS \cdot (X'X)^{-1}_{33}/(n-3)}\),
является
\begin{answerlist}
  \item смещённой
  \item несмещённой
  \item состоятельной
  \item смещённой вниз
  \item смещённой вверх
\end{answerlist}
\end{question}




\begin{question}
Инструмент \(Z_t\) для состоятельной оценки динамической модели
\(Y_{t} =\alpha +\beta_0 X_t + \beta_1 Y_{t-1} + \varepsilon_t\), где
\(\varepsilon_t = u_t + \lambda_1 \varepsilon_{t-1} + \lambda_2 \varepsilon_{t-2}\),
\begin{answerlist}[2]
  \item не требуется
  \item удовлетворяет условию \(\Corr(Z_t, u_t) \to 1\)
  \item удовлетворяет условию \(\Corr(Z_t, Y_{t-1}) \to 1\)
  \item удовлетворяет условию \(\Corr(Z_t, Y_{t-1}) =0\)
  \item удовлетворяет условию \(\Corr(Z_t, X_t) =0\)
\end{answerlist}
\end{question}






\newpage
\rfoot{Задачи}
\checktable
\putyourname


\begin{enumerate}
  \item (5 баллов) Случайные величины $X$ и $Y$ независимы и имеют хи-квадрат распределение
  с 5 и с 10 степенями свободы, соответственно. Случайная величина $Z$ равна $Z = (X+Y)/X$.

  Найдите значение $z^*$ такое, что $\P(Z > z^*)=0.05$.
  \item (5 баллов) Докажите, что для модели парной регрессии $Y_i = \beta_0 + \beta_1 X_i + \varepsilon_i$,
оцененной с помощью МНК, выполнено равенство $\sum_{i=1}^n Y_i = \sum_{i=1}^n \hat Y_i$.

  \item (5 баллов) Аккуратно сформулируйте теорему Гаусса-Маркова для случая парной регрессии.

  \item (10 баллов) На основании 62 наблюдений Чебурашка оценил функцию спроса на апельсины:

 \[
 \hat Y_i = \underset{(1.6)}{3} - \underset{(0.2)}{1.25} X_i, \text{ где } \sum_i (X_i - \bar X)^2 =2.25
 \]

 В скобках приведены стандартные ошибки коэффициентов, случайные ошибки в регрессии можно считать нормальными.


  \begin{enumerate}
    \item Проверьте гипотезы о значимости каждого из коэффициентов регрессии при уровне значимости 5\%.
    \item Проверьте гипотезу о равенстве коэффициента наклона -1 при уровне значимости 5\%
    и односторонней альтернативной гипотезе, что коэффициент наклона меньше -1.
    \item Найдите оценку дисперсии ошибок.
    \item Найдите 95\% интервальный индивидуальный прогноз в точке $X=8$.
  \end{enumerate}
\end{enumerate}


\newpage
\lfoot{Вариант $\kappa$}
\rfoot{Тест}
\setcounter{question}{0}


\putyourname

\begin{question}
При выполненных условиях регулярности оценки метода максимального
правдоподобия могут \textbf{НЕ} являться
\begin{answerlist}
  \item инвариантными
  \item асимптотически нормальными
  \item асимптотически эффективными
  \item состоятельными
  \item несмещёнными
\end{answerlist}
\end{question}




\begin{question}
Стьюдентизированные остатки регрессии используются
\begin{answerlist}
  \item на первом шаге при проведении теста Годфельда-Квандта
  \item в методе главных компонент
  \item на первом шаге двухшагового МНК
  \item в тесте Саргана
  \item для выявления выбросов
\end{answerlist}
\end{question}




\begin{question}
Рассмотрим логистическую регрессию с пятью регрессорами помимо
константы, оцениваемую методом максимального правдоподобия по \(n\)
наблюдениям. Cтатистика \(\hat \beta_3 / se(\hat\beta_3)\) для проверки
значимости коэффициента \(\beta_3\) имеет
\begin{answerlist}
  \item \(\chi^2\)-распределение с одной степенью свободы
  \item \(t\)-распределение с \(n\) степенями свободы
  \item \(t\)-распределение с \(n-6\) степенями свободы
  \item \(t\)-распределение с \(n-5\) степенями свободы
  \item асимптотически нормальное распределение
\end{answerlist}
\end{question}

\begin{solution}
========
\end{solution}



\begin{question}
Для модели \(Y_i = \beta X_i + \varepsilon_i\) c
\(\E(\varepsilon_i) = 0\) известно, что оценка
\(\hat \beta = \frac{\sum_{i=1}^n Y_i}{\sum_{i=1}^n X_i}\) обладает
наименьшей дисперсией среди линейных несмещённых оценок.

Дисперсии \(\Var(\varepsilon_i)\) пропорциональны
\begin{answerlist}
  \item \(\sqrt{X_i}\)
  \item \(X_i^2\)
  \item \(1/X_i\)
  \item \(X_i\)
  \item \(1/X_i^2\)
\end{answerlist}
\end{question}




\begin{question}
Основная гипотеза модели адаптивных ожиданий состоит в том, что
\begin{answerlist}[2]
  \item \(X_t^e - X_{t-1}^e = (1-\lambda) (X_t - X_{t-1}), \; 0 \leq \lambda < 1\)
  \item \(Y_t^e = \left(1 - \frac{1}{\delta}\right) Y_{t-1} + \frac{1}{\delta} Y_{t}, \; 0 < \delta \leq 1\)
  \item \(Y_t - Y_{t-1}^e = \delta (Y_t^e - Y_{t-1}), \; 0 < \delta \leq 1\)
  \item \(X_{t+1}^e - X_{t}^e = \lambda (X_{t} - X_{t}^e), \; 0 \leq \lambda < 1\)
  \item \(Y_t - Y_{t-1} = \delta (Y_t^e - Y_{t-1}), \; 0 < \delta \leq 1\)
\end{answerlist}
\end{question}





\newpage
\putyourname

\begin{question}
Использование скорректированных стандартных ошибок Уайта при
гомоскедастичности приведет к
\begin{answerlist}
  \item понижению эффективности МНК оценок коэффициентов
  \item смещённости МНК оценок коэффициентов
  \item несостоятельности МНК оценок коэффициентов
  \item получению состоятельной оценки дисперсии случайной ошибки
  \item повышению эффективности МНК оценок коэффициентов
\end{answerlist}
\end{question}

\begin{solution}
========
\end{solution}



\begin{question}
В линейной модели \(Y_i = \beta_0 + \beta_1 X_i + \varepsilon_i\)
стохастический регрессор и случайный член \(\varepsilon_i\)
коррелированы. Состоятельные оценки коэффициентов можно получить с
помощью
\begin{answerlist}
  \item обобщённого МНК
  \item метода наименьших квадратов
  \item взвешенного МНК
  \item метода главных компонент
  \item метода инструментальных переменных
\end{answerlist}
\end{question}




\begin{question}
В предположениях нормальности ошибок ширина 95\%-го интервала для
ожидаемого (среднего) значения \(Y_{n+1}\) равна \(1200\). Известно, что
\(\hat\sigma = 400\) и \(n=60\). Ширина 95\%-го интервала для
фактического (индивидального) значения \(Y_{n+1}\) примерно равна
\begin{answerlist}
  \item \(1500\)
  \item \(1600\)
  \item \(2000\)
  \item \(1000\)
  \item \(1400\)
\end{answerlist}
\end{question}

\begin{solution}
========
\end{solution}



\begin{question}
Рассмотрим модель
\(Y_i= \beta_0 + \beta_z Z_{i} + \beta_w W_{i} + \varepsilon\) при
гетероскедастичности. Стандартная ошибка МНК-оценки, рассчитываемая по
формуле \(se(\hat\beta_w)=\sqrt{RSS \cdot (X'X)^{-1}_{33}/(n-3)}\),
является
\begin{answerlist}
  \item смещённой
  \item несмещённой
  \item состоятельной
  \item смещённой вниз
  \item смещённой вверх
\end{answerlist}
\end{question}




\begin{question}
Инструмент \(Z_t\) для состоятельной оценки динамической модели
\(Y_{t} =\alpha +\beta_0 X_t + \beta_1 Y_{t-1} + \varepsilon_t\), где
\(\varepsilon_t = u_t + \lambda_1 \varepsilon_{t-1} + \lambda_2 \varepsilon_{t-2}\),
\begin{answerlist}[2]
  \item не требуется
  \item удовлетворяет условию \(\Corr(Z_t, u_t) \to 1\)
  \item удовлетворяет условию \(\Corr(Z_t, Y_{t-1}) \to 1\)
  \item удовлетворяет условию \(\Corr(Z_t, Y_{t-1}) =0\)
  \item удовлетворяет условию \(\Corr(Z_t, X_t) =0\)
\end{answerlist}
\end{question}






\newpage
\rfoot{Задачи}
\checktable
\putyourname


\begin{enumerate}
  \item (5 баллов) Случайные величины $X$ и $Y$ независимы и имеют хи-квадрат распределение
  с 6 и с 10 степенями свободы, соответственно. Случайная величина $Z$ равна $Z = (X+Y)/X$.

  Найдите значение $z^*$ такое, что $\P(Z > z^*)=0.05$.
  \item (5 баллов) Докажите, что для модели парной регрессии $Y_i = \beta_0 + \beta_1 X_i + \varepsilon_i$,
оцененной с помощью МНК, выполнено равенство $\sum_{i=1}^n Y_i = \sum_{i=1}^n \hat Y_i$.

  \item (5 баллов) Аккуратно сформулируйте теорему Гаусса-Маркова для случая парной регрессии.

  \item (10 баллов) На основании 52 наблюдений Чебурашка оценил функцию спроса на апельсины:

 \[
 \hat Y_i = \underset{(4.8)}{9} - \underset{(0.2)}{1.25} X_i,  \text{ где }  \sum_i (X_i - \bar X)^2 =2.25
 \]

 В скобках приведены стандартные ошибки коэффициентов, случайные ошибки в регрессии можно считать нормальными.


  \begin{enumerate}
    \item Проверьте гипотезы о значимости каждого из коэффициентов регрессии при уровне значимости 5\%.
    \item Проверьте гипотезу о равенстве коэффициента наклона -1 при уровне значимости 5\%
    и односторонней альтернативной гипотезе, что коэффициент наклона меньше -1.
    \item Найдите оценку дисперсии ошибок.
    \item Найдите 95\% интервальный индивидуальный прогноз в точке $X=9$.
  \end{enumerate}
\end{enumerate}


\newpage
\lfoot{Вариант $\delta$}
\rfoot{Тест}
\setcounter{question}{0}


\putyourname

\begin{question}
При выполненных условиях регулярности оценки метода максимального
правдоподобия могут \textbf{НЕ} являться
\begin{answerlist}
  \item инвариантными
  \item асимптотически нормальными
  \item асимптотически эффективными
  \item состоятельными
  \item несмещёнными
\end{answerlist}
\end{question}




\begin{question}
Стьюдентизированные остатки регрессии используются
\begin{answerlist}
  \item на первом шаге при проведении теста Годфельда-Квандта
  \item в методе главных компонент
  \item на первом шаге двухшагового МНК
  \item в тесте Саргана
  \item для выявления выбросов
\end{answerlist}
\end{question}




\begin{question}
Рассмотрим логистическую регрессию с пятью регрессорами помимо
константы, оцениваемую методом максимального правдоподобия по \(n\)
наблюдениям. Cтатистика \(\hat \beta_3 / se(\hat\beta_3)\) для проверки
значимости коэффициента \(\beta_3\) имеет
\begin{answerlist}
  \item \(\chi^2\)-распределение с одной степенью свободы
  \item \(t\)-распределение с \(n\) степенями свободы
  \item \(t\)-распределение с \(n-6\) степенями свободы
  \item \(t\)-распределение с \(n-5\) степенями свободы
  \item асимптотически нормальное распределение
\end{answerlist}
\end{question}

\begin{solution}
========
\end{solution}



\begin{question}
Для модели \(Y_i = \beta X_i + \varepsilon_i\) c
\(\E(\varepsilon_i) = 0\) известно, что оценка
\(\hat \beta = \frac{\sum_{i=1}^n Y_i}{\sum_{i=1}^n X_i}\) обладает
наименьшей дисперсией среди линейных несмещённых оценок.

Дисперсии \(\Var(\varepsilon_i)\) пропорциональны
\begin{answerlist}
  \item \(\sqrt{X_i}\)
  \item \(X_i^2\)
  \item \(1/X_i\)
  \item \(X_i\)
  \item \(1/X_i^2\)
\end{answerlist}
\end{question}




\begin{question}
Основная гипотеза модели адаптивных ожиданий состоит в том, что
\begin{answerlist}[2]
  \item \(X_t^e - X_{t-1}^e = (1-\lambda) (X_t - X_{t-1}), \; 0 \leq \lambda < 1\)
  \item \(Y_t^e = \left(1 - \frac{1}{\delta}\right) Y_{t-1} + \frac{1}{\delta} Y_{t}, \; 0 < \delta \leq 1\)
  \item \(Y_t - Y_{t-1}^e = \delta (Y_t^e - Y_{t-1}), \; 0 < \delta \leq 1\)
  \item \(X_{t+1}^e - X_{t}^e = \lambda (X_{t} - X_{t}^e), \; 0 \leq \lambda < 1\)
  \item \(Y_t - Y_{t-1} = \delta (Y_t^e - Y_{t-1}), \; 0 < \delta \leq 1\)
\end{answerlist}
\end{question}





\newpage
\putyourname

\begin{question}
Использование скорректированных стандартных ошибок Уайта при
гомоскедастичности приведет к
\begin{answerlist}
  \item понижению эффективности МНК оценок коэффициентов
  \item смещённости МНК оценок коэффициентов
  \item несостоятельности МНК оценок коэффициентов
  \item получению состоятельной оценки дисперсии случайной ошибки
  \item повышению эффективности МНК оценок коэффициентов
\end{answerlist}
\end{question}

\begin{solution}
========
\end{solution}



\begin{question}
В линейной модели \(Y_i = \beta_0 + \beta_1 X_i + \varepsilon_i\)
стохастический регрессор и случайный член \(\varepsilon_i\)
коррелированы. Состоятельные оценки коэффициентов можно получить с
помощью
\begin{answerlist}
  \item обобщённого МНК
  \item метода наименьших квадратов
  \item взвешенного МНК
  \item метода главных компонент
  \item метода инструментальных переменных
\end{answerlist}
\end{question}




\begin{question}
В предположениях нормальности ошибок ширина 95\%-го интервала для
ожидаемого (среднего) значения \(Y_{n+1}\) равна \(1200\). Известно, что
\(\hat\sigma = 400\) и \(n=60\). Ширина 95\%-го интервала для
фактического (индивидального) значения \(Y_{n+1}\) примерно равна
\begin{answerlist}
  \item \(1500\)
  \item \(1600\)
  \item \(2000\)
  \item \(1000\)
  \item \(1400\)
\end{answerlist}
\end{question}

\begin{solution}
========
\end{solution}



\begin{question}
Рассмотрим модель
\(Y_i= \beta_0 + \beta_z Z_{i} + \beta_w W_{i} + \varepsilon\) при
гетероскедастичности. Стандартная ошибка МНК-оценки, рассчитываемая по
формуле \(se(\hat\beta_w)=\sqrt{RSS \cdot (X'X)^{-1}_{33}/(n-3)}\),
является
\begin{answerlist}
  \item смещённой
  \item несмещённой
  \item состоятельной
  \item смещённой вниз
  \item смещённой вверх
\end{answerlist}
\end{question}




\begin{question}
Инструмент \(Z_t\) для состоятельной оценки динамической модели
\(Y_{t} =\alpha +\beta_0 X_t + \beta_1 Y_{t-1} + \varepsilon_t\), где
\(\varepsilon_t = u_t + \lambda_1 \varepsilon_{t-1} + \lambda_2 \varepsilon_{t-2}\),
\begin{answerlist}[2]
  \item не требуется
  \item удовлетворяет условию \(\Corr(Z_t, u_t) \to 1\)
  \item удовлетворяет условию \(\Corr(Z_t, Y_{t-1}) \to 1\)
  \item удовлетворяет условию \(\Corr(Z_t, Y_{t-1}) =0\)
  \item удовлетворяет условию \(\Corr(Z_t, X_t) =0\)
\end{answerlist}
\end{question}






\newpage
\rfoot{Задачи}
\checktable
\putyourname


\begin{enumerate}
  \item (5 баллов) Случайные величины $X$ и $Y$ независимы и имеют хи-квадрат распределение
  с 7 и с 10 степенями свободы, соответственно. Случайная величина $Z$ равна $Z = (X+Y)/X$.

  Найдите значение $z^*$ такое, что $\P(Z > z^*)=0.05$.
  \item (5 баллов) Докажите, что для модели парной регрессии $Y_i = \beta_0 + \beta_1 X_i + \varepsilon_i$,
оцененной с помощью МНК, выполнено равенство $\sum_{i=1}^n Y_i = \sum_{i=1}^n \hat Y_i$.

  \item (5 баллов) Аккуратно сформулируйте теорему Гаусса-Маркова для случая парной регрессии.

  \item (10 баллов) На основании 42 наблюдений Чебурашка оценил функцию спроса на апельсины:

 \[
 \hat Y_i = \underset{(0.8)}{1.5} - \underset{(0.2)}{1.25} X_i,  \text{ где }  \sum_i (X_i - \bar X)^2 =2.25
 \]

 В скобках приведены стандартные ошибки коэффициентов, случайные ошибки в регрессии можно считать нормальными.


  \begin{enumerate}
    \item Проверьте гипотезы о значимости каждого из коэффициентов регрессии при уровне значимости 5\%.
    \item Проверьте гипотезу о равенстве коэффициента наклона -1 при уровне значимости 5\%
    и односторонней альтернативной гипотезе, что коэффициент наклона меньше -1.
    \item Найдите оценку дисперсии ошибок.
    \item Найдите 95\% интервальный индивидуальный прогноз в точке $X=10$.
  \end{enumerate}
\end{enumerate}



\newpage
\lfoot{Вариант $\omega$}
\rfoot{Тест}
\setcounter{question}{0}


\putyourname

\begin{question}
При выполненных условиях регулярности оценки метода максимального
правдоподобия могут \textbf{НЕ} являться
\begin{answerlist}
  \item инвариантными
  \item асимптотически нормальными
  \item асимптотически эффективными
  \item состоятельными
  \item несмещёнными
\end{answerlist}
\end{question}




\begin{question}
Стьюдентизированные остатки регрессии используются
\begin{answerlist}
  \item на первом шаге при проведении теста Годфельда-Квандта
  \item в методе главных компонент
  \item на первом шаге двухшагового МНК
  \item в тесте Саргана
  \item для выявления выбросов
\end{answerlist}
\end{question}




\begin{question}
Рассмотрим логистическую регрессию с пятью регрессорами помимо
константы, оцениваемую методом максимального правдоподобия по \(n\)
наблюдениям. Cтатистика \(\hat \beta_3 / se(\hat\beta_3)\) для проверки
значимости коэффициента \(\beta_3\) имеет
\begin{answerlist}
  \item \(\chi^2\)-распределение с одной степенью свободы
  \item \(t\)-распределение с \(n\) степенями свободы
  \item \(t\)-распределение с \(n-6\) степенями свободы
  \item \(t\)-распределение с \(n-5\) степенями свободы
  \item асимптотически нормальное распределение
\end{answerlist}
\end{question}

\begin{solution}
========
\end{solution}



\begin{question}
Для модели \(Y_i = \beta X_i + \varepsilon_i\) c
\(\E(\varepsilon_i) = 0\) известно, что оценка
\(\hat \beta = \frac{\sum_{i=1}^n Y_i}{\sum_{i=1}^n X_i}\) обладает
наименьшей дисперсией среди линейных несмещённых оценок.

Дисперсии \(\Var(\varepsilon_i)\) пропорциональны
\begin{answerlist}
  \item \(\sqrt{X_i}\)
  \item \(X_i^2\)
  \item \(1/X_i\)
  \item \(X_i\)
  \item \(1/X_i^2\)
\end{answerlist}
\end{question}




\begin{question}
Основная гипотеза модели адаптивных ожиданий состоит в том, что
\begin{answerlist}[2]
  \item \(X_t^e - X_{t-1}^e = (1-\lambda) (X_t - X_{t-1}), \; 0 \leq \lambda < 1\)
  \item \(Y_t^e = \left(1 - \frac{1}{\delta}\right) Y_{t-1} + \frac{1}{\delta} Y_{t}, \; 0 < \delta \leq 1\)
  \item \(Y_t - Y_{t-1}^e = \delta (Y_t^e - Y_{t-1}), \; 0 < \delta \leq 1\)
  \item \(X_{t+1}^e - X_{t}^e = \lambda (X_{t} - X_{t}^e), \; 0 \leq \lambda < 1\)
  \item \(Y_t - Y_{t-1} = \delta (Y_t^e - Y_{t-1}), \; 0 < \delta \leq 1\)
\end{answerlist}
\end{question}





\newpage
\putyourname

\begin{question}
Использование скорректированных стандартных ошибок Уайта при
гомоскедастичности приведет к
\begin{answerlist}
  \item понижению эффективности МНК оценок коэффициентов
  \item смещённости МНК оценок коэффициентов
  \item несостоятельности МНК оценок коэффициентов
  \item получению состоятельной оценки дисперсии случайной ошибки
  \item повышению эффективности МНК оценок коэффициентов
\end{answerlist}
\end{question}

\begin{solution}
========
\end{solution}



\begin{question}
В линейной модели \(Y_i = \beta_0 + \beta_1 X_i + \varepsilon_i\)
стохастический регрессор и случайный член \(\varepsilon_i\)
коррелированы. Состоятельные оценки коэффициентов можно получить с
помощью
\begin{answerlist}
  \item обобщённого МНК
  \item метода наименьших квадратов
  \item взвешенного МНК
  \item метода главных компонент
  \item метода инструментальных переменных
\end{answerlist}
\end{question}




\begin{question}
В предположениях нормальности ошибок ширина 95\%-го интервала для
ожидаемого (среднего) значения \(Y_{n+1}\) равна \(1200\). Известно, что
\(\hat\sigma = 400\) и \(n=60\). Ширина 95\%-го интервала для
фактического (индивидального) значения \(Y_{n+1}\) примерно равна
\begin{answerlist}
  \item \(1500\)
  \item \(1600\)
  \item \(2000\)
  \item \(1000\)
  \item \(1400\)
\end{answerlist}
\end{question}

\begin{solution}
========
\end{solution}



\begin{question}
Рассмотрим модель
\(Y_i= \beta_0 + \beta_z Z_{i} + \beta_w W_{i} + \varepsilon\) при
гетероскедастичности. Стандартная ошибка МНК-оценки, рассчитываемая по
формуле \(se(\hat\beta_w)=\sqrt{RSS \cdot (X'X)^{-1}_{33}/(n-3)}\),
является
\begin{answerlist}
  \item смещённой
  \item несмещённой
  \item состоятельной
  \item смещённой вниз
  \item смещённой вверх
\end{answerlist}
\end{question}




\begin{question}
Инструмент \(Z_t\) для состоятельной оценки динамической модели
\(Y_{t} =\alpha +\beta_0 X_t + \beta_1 Y_{t-1} + \varepsilon_t\), где
\(\varepsilon_t = u_t + \lambda_1 \varepsilon_{t-1} + \lambda_2 \varepsilon_{t-2}\),
\begin{answerlist}[2]
  \item не требуется
  \item удовлетворяет условию \(\Corr(Z_t, u_t) \to 1\)
  \item удовлетворяет условию \(\Corr(Z_t, Y_{t-1}) \to 1\)
  \item удовлетворяет условию \(\Corr(Z_t, Y_{t-1}) =0\)
  \item удовлетворяет условию \(\Corr(Z_t, X_t) =0\)
\end{answerlist}
\end{question}






\newpage
\rfoot{Задачи}
\checktable
\putyourname


\begin{enumerate}
  \item (5 баллов) Случайные величины $X$ и $Y$ независимы и имеют хи-квадрат распределение
  с 8 и с 10 степенями свободы, соответственно. Случайная величина $Z$ равна $Z = (X+Y)/X$.

  Найдите значение $z^*$ такое, что $\P(Z > z^*)=0.05$.
  \item (5 баллов) Докажите, что для модели парной регрессии $Y_i = \beta_0 + \beta_1 X_i + \varepsilon_i$,
оцененной с помощью МНК, выполнено равенство $\sum_{i=1}^n Y_i = \sum_{i=1}^n \hat Y_i$.

  \item (5 баллов) Аккуратно сформулируйте теорему Гаусса-Маркова для случая парной регрессии.

  \item (10 баллов) На основании 32 наблюдений Чебурашка оценил функцию спроса на апельсины:

 \[
 \hat Y_i = \underset{(3.2)}{6} - \underset{(0.2)}{1.25} X_i,  \text{ где }  \sum_i (X_i - \bar X)^2 =2.25
 \]

  В скобках приведены стандартные ошибки коэффициентов, случайные ошибки в регрессии можно считать нормальными.

  \begin{enumerate}
    \item Проверьте гипотезы о значимости каждого из коэффициентов регрессии при уровне значимости 5\%.
    \item Проверьте гипотезу о равенстве коэффициента наклона -1 при уровне значимости 5\%
    и односторонней альтернативной гипотезе, что коэффициент наклона меньше -1.
    \item Найдите оценку дисперсии ошибок.
    \item Найдите 95\% интервальный индивидуальный прогноз в точке $X=11$.
  \end{enumerate}
\end{enumerate}





\end{document}
