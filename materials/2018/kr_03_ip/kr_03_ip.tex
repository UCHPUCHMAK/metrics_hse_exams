\documentclass[12pt]{article}

\usepackage{tikz} % картинки в tikz
\usepackage{microtype} % свешивание пунктуации

\usepackage{array} % для столбцов фиксированной ширины

\usepackage{indentfirst} % отступ в первом параграфе

\usepackage{sectsty} % для центрирования названий частей
\allsectionsfont{\centering}

\usepackage{amsmath} % куча стандартных математических плюшек

\usepackage{comment}
\usepackage{amsfonts}

\usepackage[top=2cm, left=1cm, right=1cm, bottom=2cm]{geometry} % размер текста на странице

\usepackage{lastpage} % чтобы узнать номер последней страницы

\usepackage{enumitem} % дополнительные плюшки для списков
%  например \begin{enumerate}[resume] позволяет продолжить нумерацию в новом списке
\usepackage{caption}

\usepackage{hyperref} % гиперссылки

\usepackage{multicol} % текст в несколько столбцов


\usepackage{fancyhdr} % весёлые колонтитулы
\pagestyle{fancy}
\lhead{Максимум}
\chead{2019-03-30}
\rhead{:)}
\lfoot{}
\cfoot{}
\rfoot{}
\renewcommand{\headrulewidth}{0.4pt}
\renewcommand{\footrulewidth}{0.4pt}





\usepackage{todonotes} % для вставки в документ заметок о том, что осталось сделать
% \todo{Здесь надо коэффициенты исправить}
% \missingfigure{Здесь будет Последний день Помпеи}
% \listoftodos --- печатает все поставленные \todo'шки


% более красивые таблицы
\usepackage{booktabs}
% заповеди из докупентации:
% 1. Не используйте вертикальные линни
% 2. Не используйте двойные линии
% 3. Единицы измерения - в шапку таблицы
% 4. Не сокращайте .1 вместо 0.1
% 5. Повторяющееся значение повторяйте, а не говорите "то же"


\usepackage{fontspec}
\usepackage{polyglossia}

\setmainlanguage{russian}
\setotherlanguages{english}

% download "Linux Libertine" fonts:
% http://www.linuxlibertine.org/index.php?id=91&L=1
\setmainfont{Linux Libertine O} % or Helvetica, Arial, Cambria
% why do we need \newfontfamily:
% http://tex.stackexchange.com/questions/91507/
\newfontfamily{\cyrillicfonttt}{Linux Libertine O}

\AddEnumerateCounter{\asbuk}{\russian@alph}{щ} % для списков с русскими буквами
\setlist[enumerate, 2]{label=\asbuk*),ref=\asbuk*}

%% эконометрические сокращения
\DeclareMathOperator{\Cov}{Cov}
\DeclareMathOperator{\Corr}{Corr}
\DeclareMathOperator{\Var}{Var}
\DeclareMathOperator{\E}{E}
\def \hb{\hat{\beta}}
\def \hs{\hat{\sigma}}
\def \htheta{\hat{\theta}}
\def \s{\sigma}
\def \hy{\hat{y}}
\def \hY{\hat{Y}}
\def \v1{\vec{1}}
\def \e{\varepsilon}
\def \he{\hat{\e}}
\def \z{z}
\def \hVar{\widehat{\Var}}
\def \hCorr{\widehat{\Corr}}
\def \hCov{\widehat{\Cov}}
\def \cN{\mathcal{N}}
\def \P{\mathbb{P}}


\begin{document}


\fbox{
  \begin{minipage}{42em}
    Имя, фамилия и номер группы:\vspace*{3ex}\par
    \noindent\dotfill\vspace{2mm}
  \end{minipage}
}

\vspace{10pt}

Ровно 228 лет назад, 30 марта 1791 Национальное собрание Франции ввело определение метра: одна сорокамиллионная часть длины парижского меридиана.


\begin{enumerate}
\item Рассмотрим логит-модель $\P(y_i = 1) = \Lambda(X_{i.}\beta)$, где $X_{i.}$ — $i$-ая строчка матрицы
регрессоров. Людовик XIV знает, что оценки логит-модели в явном виде аналитически не считаются. Поэтому он использует две технологии.

Технология 1. Разложить лог-функцию правдоподобия в ряд Тейлора до членов второго порядка и максимизировать полученную функцию.

Технология 2. Стартуя из точки $\beta = 0$ сделать один шаг равный градиенту лог-функции правдоподобия.


\begin{enumerate}
  \item Помогите Людовику получить обе аппроксимации оценок логит-модели.
  \item С каким известным алгоритмом совпадает одна из этих оценок?
\end{enumerate}


\item Храбрый исследователь Шарль Ожье́ де Бац де Кастельмо́р, граф д'Артанья́н, оценивает модель $y_i = \beta x_i + u_i$ по трём наблюдениям.
Ошибки $u_i$ имеют многомерное нормальное распределение с нулевым ожиданием и ковариационной матрицей
$\sigma^2\begin{pmatrix}
1 & 0 & 1 \\
0 & 1 & 1 \\
1 & 1 & 2 \\
\end{pmatrix}$.

Наблюдения известны:

\begin{tabular}{cccc}
\toprule
$y_i$ & 1 & 2 & 1  \\
$x_i$ & 0 & 3 & -1 \\
\bottomrule
\end{tabular}


\begin{enumerate}
  \item Найдите наиболее эффективную оценку в классе линейных по игреку несмещённых оценок.
  \item Какой уровень доверия обеспечивает интервал $[\hb -1;\hb +1]$?
  \item Постройте 95\%-й предиктивный интервал для $y_4$, если известно, что $u_4$ не зависит
  от предыдущих ошибок и имеет дисперсию $3\sigma^2$.
\end{enumerate}


\item Монетку подбросили 100 раз. Мария Антуанетта помнит, что в первых 70-и бросках было 30 орлов.
А графиня де Полиньяк помнит, что в последних 60-и бросках было 20 орлов.


\begin{enumerate}
  \item Постройте GMM оценку вероятности орла с единичной взвешивающей матрицей.
  \item Постройте оптимальную GMM оценку.
  \item Постройте 95\% доверительный интервал, используя каждую из оценок
\end{enumerate}

\newpage
\item Подарок для тех, кто прорешал прошлогодний вариант :)


   Исследовательница Несмеяна вывела хитрую формулу для $\hat a$ — несмещённой оценки неизвестного векторного параметра $a$.
    Обозначим $s(a)$ — вектор-столбец градиент логарифмической функции правдоподобия. Докажите, что для оценки Несмеяны выполнено неравенство Крамера-Рао,
    а именно, что матрица $M$  положительна определена.

\[
M = \Var(s(a))\cdot \Var(\hat a) - I_{k\times k}.
\]


Подсказки:

    \begin{enumerate}
      \item Вспомните, чему равно $\E(s(a))$. Достаточно просто вспомнить, доказывать не требуется!
      \item Найдите скаляры $\Cov\left(\hat a_1, \frac{\partial \ell}{\partial a_1}\right)$,
	$\Cov\left(\hat a_1, \frac{\partial \ell}{\partial a_2}\right)$
	и матрицу $\Cov\left(\hat a, s(a) \right)$.
      \item Рассмотрим два произвольных случайных вектора $R$ и $S$ и два вектора констант подходящей длины $\alpha$ и $\beta$.
	Найдите минимум функции $f(\alpha, \beta) = \Var(\alpha^T R + \beta^T S)$ по $\beta$.
	Выпишите явно $\beta^*(\alpha)$ и $f^*(\alpha)$.
      \item Докажите, что для произвольных случайных векторов положительно определена матрица
	\[
          \Var(R) - \Cov(R, S) \Var^{-1}(S)\Cov(S, R)
	\]
      \item Завершите доказательство векторного неравенства Крамера-Рао.


    \end{enumerate}

  Без угрызений совести можно храбро переставлять интегралы и производные :)


  \item
  Докажите, что в методе главных компонент с масштабированием переменных средняя величина $R^2$ по всем парным
  регрессиям исходных переменных на первую главную компоненту равна наибольшему сингулярному значению
  матрицы исходных переменных.

\item Докажите закон больших чисел в форме Бернштайна.

Если величины $y_1$, $y_2$, \ldots, $y_n$ имеют одинаковое ожидание, $\E(y_i) = \mu$, ограниченную дисперсию и $\Cov(y_j, y_j) \to 0$ при $|i - j|\to \infty$, то $\bar y$ сходится по вероятности к $\mu$.

\end{enumerate}

\end{document}
